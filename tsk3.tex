\documentclass[a4paper, 14pt]{report}
\usepackage[english, russian]{babel}
\usepackage[T2A]{fontenc}  % кодировка шрифта (кириллица)
\usepackage{graphicx}  % для работы с графикой
\graphicspath{{./images/}}
\usepackage[utf8]{inputenc}  % подключает UTF-8
\usepackage{setspace}  % интервалы
\onehalfspacing
\usepackage[left=30mm, top=20mm, right=10mm, bottom=20mm, nohead, footskip=10mm]{geometry}  % отступы
\usepackage[fontsize=14pt]{scrextend}  % размер текста
\usepackage[backend=biber,bibencoding=utf8,sorting=nty,maxcitenames=2,style=numeric-comp]{biblatex}  % список литературы
\addbibresource{bibliography.bib}

\begin{document}
	\chapter{Аппаратные средства вычислительной техники (АСВТ)}
	\section{Архитектура процессоров}
	Архитектура процессоров является ключевым элементом в аппаратных средствах вычислительной техники. Рассмотрение этой главы включает в себя описание основных компонентов процессора, таких как регистры, арифметико-логическое устройство, устройство управления и кэш-память. Главное внимание уделяется архитектуре операционной системы и ее взаимодействию с аппаратными средствами.(\ref{cuda2})
	
	\begin{figure}[h]
		\centering
		\includegraphics[scale=0.7]{cuda}
		\caption{Продукция от Nvidia}
		\label{cuda2}
	\end{figure}

	 Формула для вычисления скалярного произведения двух векторов X и Y с применением CUDA:(\ref{form2})
	
	\begin{equation}
		sum = 0for i = 1 to N
    sum += X[i] * Y[i]return sum
		\label{form2}
	\end{equation}
	
	
	
	
	
\end{document}